%     This is the LaTeX source file for the Pre-Course Handout for the 
%     git@github.com:brfitzpatrick/Intro_to_R 
%     Short Course
%     Copyright (C) 2015  Ben R. Fitzpatrick.
%
%    This program is free software: you can redistribute it and/or modify
%    it under the terms of the GNU General Public License as published by
%    the Free Software Foundation, either version 3 of the License, or
%    (at your option) any later version.
%
%    This program is distributed in the hope that it will be useful,
%    but WITHOUT ANY WARRANTY; without even the implied warranty of
%    MERCHANTABILITY or FITNESS FOR A PARTICULAR PURPOSE.  See the
%    GNU General Public License for more details.
%
%    You should have received a copy of the GNU General Public License
%    along with this program.  If not, see <http://www.gnu.org/licenses/>.
%
%    The course author may be contacted by email at 
%    <ben.r.fitzpatrick@gmail.com>

\documentclass{article}[12pt]
\usepackage{multirow}
\usepackage{hyperref}
\usepackage{textcomp}
\usepackage{upquote}

\begin{document}

\title{Introduction to R}
\author{Ben R. Fitzpatrick\\ 
\small PhD Candidate, Mathematical Sciences School,\\
\small Queensland University of Technology, Brisbane, QLD.}
\date{\today}
\maketitle

\section*{Prior to Course}
Please do you best to install the software listed below on the laptop you will bring to the course.
Please also ensure you have `Eduroam' \url{http://www.eduroam.edu.au/} configured on your laptop so that you can access the internet during the course.  
Internet access on the laptop you are using will be vital to completing the module on version control with GitHub.
If possible please bring some along some data that interest you, after the initial training in the visualisation software you will have the opportunity to practise visualising your data with the course demonstrator on hand to help you if you run into problems.
\newline
\newline
\textbf{Prior to the course please}: \begin{itemize}
\item download and install \textbf{R}
\item download and install \textbf{RStudio}
\item download and install the R packages `ggplot2', `rgl', `rgdal', `raster'
\item download and install \textbf{Git}
\item create a GitHub account. \end{itemize}

\textbf{Note:} Please ensure that all the listed software (R, RStudio and the R packages) are installed to the hard drive of the laptop you will bring to the course (rather than to your personal folder on your university's network shared drives).

\subsection*{Downloading and Installing R}
R is available for free download for Windows, MacOS, and Linux from the Comprehensive R Archive Network here: \url{http://cran.r-project.org/}.  At the time of writing the current release was R version 3.2.0.
\newline
\newline
Please also install the following R packages:  \begin{itemize}
 \item `ggplot2'
 \item `rgl' \end{itemize}
Install R packages with the `Tools' $\rightarrow$ `Install Packages' in RStudio or by executing the following code at the R command line (your computer will need to be connected to the internet to install R pacakges)o:
\begin{verbatim}
install.packages('ggplot2')
install.packages('rgl')
\end{verbatim}
You will need to selected a mirror to download the packages from Canberra or Melbourne are good choices if you are in Australia.

\subsection*{Downloading and Installing RStudio}
RStudio is available to download from here: \url{http://www.rstudio.com/}.
Please download the free, open source, desktop edition.
There is a good video introducing RStudio here \url{http://www.rstudio.com/products/rstudio/}.


\subsection*{Downloading and R Packages}
Prior to the course please install the R packages: `ggplot2' and `rgl'.
To install R packages with the RStudio R package manager please: \begin{enumerate}
\item ensure your computer is connected to the internet and logged in such that you can access the broader internet not just your university's intranet e.g. make sure you can visit the abc.net.au
\item open RStudio
\item from the 'Tools' menu up the top selected 'Install Packages'
\item Type the names of the packages you want to install into the dialogue box separated by commas
\item Click Install
\



\subsection*{Downloading and Installing Git}
\textbf{MS Windows} \& \textbf{Mac OS X} users please visit \url{http://git-scm.com/downloads} and follow the instructions there.\newline
\newline
\textbf{GNU+Linux} users:\\ 
Debian/Ubuntu: \begin{verbatim} sudo apt-get install git-core \end{verbatim}
Fedora/RedHat: \begin{verbatim} sudo yum install git-core \end{verbatim}

\subsection*{Creating a GitHub Account}
Navigate to \url{https://github.com/} and click the green `Sign Up' button in the top right.
\newline
\newline
\textbf{Note}: with a .edu email address you can get 5 private repositories (for now think folders to put things in) for free.
Please visit \url{https://education.github.com/discount_requests/new} to request your 5 free private repositories.
\clearpage




\section*{Optional Pre-reading}
\subsection*{Concerning R and its growing popularity}
If you'd like to find out a bit more about R and its increasing popularity I heartily refer you to the following pieces: \begin{itemize}
\item \url{http://www.revolutionanalytics.com/r-is-still-hot}
\item \url{http://www.wired.com/2015/01/microsoft-acquires-open-source-data-science-company-revolution-analytics}
\end{itemize}

\subsection*{Concerning the RStudio IDE for R}
If you'd like a little preview of the Integrated Development Environment we will be using to author R code and interact with the R process I refer you to the following sources on RStudio and its intuitive yet feature rich approach to authroing R Code and running an R process: \begin{itemize}
\item \url{https://vimeo.com/97166163}
\item \url{http://www.rstudio.com/products/rstudio/features/}
\end{itemize}

\end{document}
