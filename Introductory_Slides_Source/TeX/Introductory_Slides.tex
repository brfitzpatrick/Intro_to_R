
\documentclass[xcolor=dvipsnames]{beamer} 
\usecolortheme[named=Blue]{structure} 
\usetheme[height=10.5mm]{Rochester} 
\setbeamertemplate{items}[ball] 
\setbeamertemplate{blocks}[rounded][shadow=true] 
\setbeamertemplate{navigation symbols}{} 
\usepackage{bm}
\usepackage{rotating}
\usepackage{graphicx}
\usepackage{multirow}
\usepackage{hyperref}
\usepackage{textcomp}
\usepackage{upquote}
\usepackage[absolute,overlay]{textpos}
\newenvironment{reference}[2]{%
  \begin{textblock*}{\textwidth}(#1,#2)
      \footnotesize\it\bgroup\color{red!50!black}}{\egroup\end{textblock*}}
%\graphicspath{ {/home/ben/PhD/Armidale_Updates/2014_03_14/Figures/} }
\begin{document}

\begin{frame} %1
% \frametitle{Addressing Common Challenges in Spatial Modeling of Ecologies and Environments}
%\begin{center}
\textbf{\huge Introduction to R}\\
a language and environment for statistical computing and graphics %
%\end{center}

\begin{figure}
\includegraphics[width = 0.35\textwidth]{/home/ben/Intro_to_R/Introductory_Slides_Source/Images/R_logo.png}
\end{figure}
\small Ben R. Fitzpatrick\\
\tiny PhD Candidate, Statistical Science, Mathematical Sciences School, Queensland University of Technology
\newline
\begin{columns}
\begin{column}{3cm}
\tiny 0000-0003-1916-0939
\end{column}
\begin{column}{3cm}
\tiny github.com/brfitzpatrick/
\end{column}
\begin{column}{3cm}
\tiny @benrfitzpatrick
\end{column}
\end{columns}
\end{frame}

\begin{frame} 
\frametitle{What is R?}
\end{frame}

\begin{frame} 
\frametitle{Why Use R?}
\framesubtitle{R is Free in the Sense of Free Speach \& Free Beer}
\end{frame}

\begin{frame} 
\frametitle{Why Use R?}
\framesubtitle{R has many packages implementing a broad range of statistical analyses}

\begin{figure}
\includegraphics[height = 0.95\textheight]{/home/ben/Intro_to_R/Introductory_Slides_Source/Images/CRAN_Task_Views.png}
\end{figure}

\end{frame}

\begin{frame} 
\frametitle{Why Use R?}
\framesubtitle{R has powerful graphic authoring capabilities}

\begin{figure}
\includegraphics[width = 0.95\textwidth]{/home/ben/Intro_to_R/Introductory_Slides_Source/Images/179.png}
\end{figure}

\tiny 3D visualisation produced with the `rgl' R package

\end{frame}

\begin{frame} 
\frametitle{Why Use R?}
\framesubtitle{R has powerful graphic authoring capabilities}

\begin{figure}
\includegraphics[width = 0.95\textheight]{/home/ben/Intro_to_R/Introductory_Slides_Source/Images/desert.pdf}
\end{figure}

\tiny Geospatial Visualisation produced with the R packages `raster' \& `ggplot2'

\end{frame}




\begin{frame} 
\frametitle{Why Use R?}
\framesubtitle{R is Popular with Large \& Steadily Growing User Base}


\end{frame}


\begin{frame} 
\frametitle{Why Use R?}
\framesubtitle{Popular}
\end{frame}



\begin{frame} 
\frametitle{Ways to Use R}
\begin{itemize}
\item via a termial on the command line
\item via the default clients for MS Windows \& Mac OS
\item via one of the many Integrated Development Environments \begin{itemize}
 \item RStudio
 \item Tinn-R
 \item Emacs Speaks Statistics \end{itemize}
\end{itemize}
\end{frame}

\begin{frame}
\frametitle{Ways to Use R:}
\framesubtitle{In a termial e.g. on MS Windows}
%\begin{figure}
%\includegraphics[width = 0.35\textwidth]{~/Presents_Intro_to_R/Introductory_Slides_Source/Images/R_Console_MacOS.png}
%\end{figure}
\end{frame}

\begin{frame}
\frametitle{Ways to Use R:}
\framesubtitle{In a termial e.g. on Mac OS}
\begin{figure}
\includegraphics[width = \textwidth]{/home/ben/Intro_to_R/Introductory_Slides_Source/Images/R_Console_MacOS.png}
\end{figure}
\end{frame}

\begin{frame}
\frametitle{Ways to Use R:}
\framesubtitle{In a termial e.g. on GNU+Linux}
\begin{figure}
\includegraphics[width = \textwidth]{/home/ben/Intro_to_R/Introductory_Slides_Source/Images/R_in_Ubuntu_Terminal_2.png}
\end{figure}
\end{frame}

\begin{frame}
\frametitle{Ways to Use R:}
\framesubtitle{Default Windows Client}
\begin{figure}
\includegraphics[width = \textwidth]{/home/ben/Intro_to_R/Introductory_Slides_Source/Images/R_on_Windows.png}
\end{figure}
\end{frame}

\begin{frame}
\frametitle{Ways to Use R:}
\framesubtitle{Tinn-R Integrated Development Environement}
\begin{figure}
\includegraphics[width = \textwidth]{/home/ben/Intro_to_R/Introductory_Slides_Source/Images/TinnR.png}
\end{figure}
\end{frame}

\begin{frame}
\frametitle{Ways to Use R:}
\framesubtitle{Emacs Speaks Statistics Integrated Development Environement}
\begin{figure}
\includegraphics[width = \textwidth]{/home/ben/Intro_to_R/Introductory_Slides_Source/Images/R_ESS_Debian_GNU_Linux.png}
\end{figure}
\end{frame}

\begin{frame}
\frametitle{Ways to Use R:}
\framesubtitle{RStudio Integrated Development Environement}
\begin{figure}
\includegraphics[width = \textwidth]{/home/ben/Intro_to_R/Introductory_Slides_Source/Images/RStudio_R.png}
\end{figure}
\end{frame}

\begin{frame}
\frametitle{For this course we'll use RStudio}
Because it's comparatively :...
\begin{itemize}
\item intuitive and easy to learn
\item feature rich
\item available for most major operating systems (MS Windows, Mac OS, various flavours of GNU+Linux)
\item interacts with Git for version control via GitHub
\end{itemize}

\end{frame}

\begin{frame}
\frametitle{The Plan}
\framesubtitle{Feel free to use this time to pursue something that interests you}
Course organised into 5 instructory modules and one extended, collaborative exercise.
\newline
\newline
Module: \begin{enumerate}
\item Introduction to R \& RStudio  
\item Graphics with the R package `ggplot2'
\item Linear Modelling in R
\item Programming in R
\item Version Control for solo \& collaborative source code management with Git \& GitHub
\item Collaborative Exercise
\end{enumerate}

\end{frame}

\begin{frame}
\frametitle{Module 1}
\framesubtitle{Introduction to R \& RStudio}
\begin{block}{Key Learning Outcomes}
Familiarisation with \begin{itemize}
\item Command Line Computing
\item RStudio Integrated Development Environment
\item Commands and arguments
\item Common Object Classes in R
\item Assigning values to Objects
\item Saving \& Loading R Workspaces
\item R Base Graphics
\item Data Input
\end{itemize}
\end{block}
\end{frame}

\begin{frame}[fragile]
\frametitle{Module 2}
\framesubtitle{Graphics with `ggplot2'}
\begin{block}{Key Learning Outcomes}
The key concepts \& mechanics of the plotting with the Grammar of Graphics$^1$ inspired `ggplot2': \begin{itemize}
\item the mechanics of \begin{verbatim} ggplot( ) \end{verbatim} command
\item the concept of aesthetic mapping
\item plotting geometries
\item scales
\item faceting
\item saving plots
\end{itemize}
\end{block}

\tiny $^1$ Leland Wilkinson, \textit{The Grammar of Graphics}, Statistics and Computing. Springer, 2nd edition, 2005.
\end{frame}

\begin{frame}
\frametitle{Module 3}
\framesubtitle{Linear Modelling in R}
\begin{block}{Key Learning Outcomes}
\begin{itemize}
\item read data into R from an external file
\item fit linear regression models
\item produce \& examine model diagnostics
\item plot data along with predictions of model and associated uncertainty
\item stepwise variable selection
\item summary statistics for model
\end{itemize}
\end{block}
\end{frame}

\begin{frame}
\frametitle{Module 4}
\framesubtitle{Programming in R}
\begin{block}{Key Learning Outcomes}
Writing:
\begin{itemize}
\item conditional statements
\item loops
\item functions
\end{itemize}
\end{block}
\end{frame}

\begin{frame}
\frametitle{Module 5}
\framesubtitle{Version Control with Git \& GitHub}
\begin{block}{Key Learning Outcomes}
\begin{itemize}
\item Motivations for managaing a coding project via a version control system
\item fundamentals of Git \& GitHub: \begin{itemize}
  \item local and remote repositories
  \item developing multiple versions of the same file
  \item combining disparate versions of the same file
  \item returning to previous version of a file without loosing the current version
  \item collaboratively editing files
  \end{itemize}
\end{itemize}
\end{block}
\end{frame}

\begin{frame}
\frametitle{Module 6}
\framesubtitle{Collaborative Exercise}
\begin{block}{The Plan}
Form small groups and collaboratively explore and analyse some data that interests you.
This could be data from one of your projects or some freely available data from the internet.
Please re-use as much of the code from the preceeding exercises as you would like to.
\end{block}

\begin{block}{Key Learning Outcomes}
Practise and in doing so consolidate the skills you have learned over this course
\end{block}
\end{frame}

\begin{frame}
\frametitle{If you're already familiar with R}
\framesubtitle{Feel free to use this time to pursue something that interests you}
You could:
\begin{itemize}
\item Visit the GitHub directory for this course and pick a code file you like to start working through \url{https://github.com/brfitzpatrick/Intro_to_R}
\item See how far you can get through the incrementally harder maths/programming problems at \url{https://projecteuler.net/}
\item pursue your own project work
\end{itemize}
tuning in occasionally for the sections that interest you.
\newline
\newline
I'll need to focus on delivering the course but I'll try to check in with you peridocially throughout the next 2.5 days.
\end{frame}

\begin{frame} 
\frametitle{Image Credits}
R Foundation, from http://www.r-project.org - Originally from http://developer.r-project.org/Logo/Rlogo.svg, modified to simpler SVG format.
%
\end{frame}


\end{document}