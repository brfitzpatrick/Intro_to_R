\documentclass[xcolor=dvipsnames]{beamer} 
\usecolortheme[named=Blue]{structure} 
\usetheme[height=10.5mm]{Rochester} 
\setbeamertemplate{items}[ball] 
\setbeamertemplate{blocks}[rounded][shadow=true] 
\setbeamertemplate{navigation symbols}{} 
\usepackage{bm}
\usepackage{rotating}
\usepackage{graphicx}
\usepackage{multirow}
\usepackage{hyperref}
\usepackage{textcomp}
\usepackage{upquote}
\usepackage[absolute,overlay]{textpos}
\newenvironment{reference}[2]{%
  \begin{textblock*}{\textwidth}(#1,#2)
      \footnotesize\it\bgroup\color{red!50!black}}{\egroup\end{textblock*}}
%\graphicspath{ {/home/ben/PhD/Armidale_Updates/2014_03_14/Figures/} }
\begin{document}

%\begin{frame} %1
% \frametitle{Addressing Common Challenges in Spatial Modeling of Ecologies and Environments}
%\begin{center}
%\textbf{\huge Concluding this Introduction to R}\\
%a language and environment for statistical computing and graphics %
%\end{center}

%\begin{figure}
%\includegraphics[width = 0.35\textwidth]{/home/ben/Intro_to_R/Introductory_Slides_Source/Images/R_logo.png}
%\end{figure}
%\small Ben R. Fitzpatrick\\
%\tiny PhD Candidate, Statistical Science, Mathematical Sciences School, Queensland University of Technology
%\newline
%\begin{columns}
%\begin{column}{3cm}
%\tiny 0000-0003-1916-0939
%\end{column}
%\begin{column}{3cm}
%\tiny github.com/brfitzpatrick/
%\end{column}
%\begin{column}{3cm}
%\tiny @benrfitzpatrick
%\end{column}
%\end{columns}
%\end{frame}

\begin{frame}
\frametitle{Summing Up}
\begin{block}{Why use R?}
Revolution Analytics introduce R in 93 sec: 
\newline
\newline
\url{https://youtu.be/TR2bHSJ_eck}
\end{block}

\begin{block}{What can R do?}
Lots!:
\newline
\newline
\url{http://cran.r-project.org/web/views/}
\end{block}

\begin{block}{Why use RStudio to use R}
Key Features of the RStudio IDE: 
\newline
\newline
\url{https://vimeo.com/97166163}
\end{block}

\end{frame}

\begin{frame}
\frametitle{Module 1}
\framesubtitle{Introduction to R \& RStudio}
\begin{block}{Key Learning Outcomes}
Familiarisation with \begin{itemize}
\item Command Line Computing
\item RStudio Integrated Development Environment
\item Commands \& arguments
\item Common Object Classes in R
\item Assigning values to Objects
\item Saving \& Loading R Workspaces
\item R Base Graphics
\item Data Input
\end{itemize}
\end{block}
\end{frame}

\begin{frame}[fragile]
\frametitle{Module 2}
\framesubtitle{Graphics with `ggplot2'}
\begin{block}{Key Learning Outcomes}
The key concepts \& mechanics of the plotting with the Grammar of Graphics$^1$ inspired `ggplot2': \begin{itemize}
\item the mechanics of the \begin{verbatim} ggplot( ) \end{verbatim} command
\item the concept of aesthetic mapping
\item plotting geometries
\item scales
\item faceting
\item saving plots
\end{itemize}
\end{block}

\tiny $^1$ Leland Wilkinson, \textit{The Grammar of Graphics}, Statistics and Computing. Springer, 2nd edition, 2005.
\end{frame}

\begin{frame}
\frametitle{Module 3}
\framesubtitle{Linear Modelling in R}
\begin{block}{Key Learning Outcomes}
\begin{itemize}
\item read data into R from an external file
\item fit linear regression models
\item produce \& examine model diagnostics
\item plot data along with predictions of model and associated uncertainty
\item conduct stepwise variable selection
\item produce summary statistics for model
\end{itemize}
\end{block}
\end{frame}

\begin{frame}
\frametitle{Module 4}
\framesubtitle{Programming in R}
\begin{block}{Key Learning Outcomes}
Writing:
\begin{itemize}
\item conditional statements
\item loops
\item functions
\newline
\end{itemize}
Solving problems by writing programs
\end{block}
\end{frame}

\begin{frame}
\frametitle{Module 5}
\framesubtitle{Version Control with Git \& GitHub}
\begin{block}{Key Learning Outcomes}
Understand:
\begin{itemize}
\item motivations for managaing a coding project via a version control system
\item fundamentals of Git \& GitHub: \begin{itemize}
  \item local and remote repositories
  \item developing multiple versions of the same file
  \item combining disparate versions of the same file
  \item returning to previous version of a file without loosing the current version
  \item collaboratively editing files
  \end{itemize}
\end{itemize}
\end{block}
\end{frame}


\begin{frame}
\frametitle{Resources for learning R}

\begin{block}{Free Courses on R}
\begin{itemize}
\item \url{http://www.lynda.com/R-tutorials/R-Statistics-Essential-Training/142447-2.html}
\newline
\item \url{https://www.coursera.org/course/rprog}
\newline
\item \url{https://www.coursera.org/course/compdata}
\end{itemize}
\end{block}

\begin{block}{Documentation the CRAN Website}
Official documentation: \begin{itemize}
\item \url{http://cran.r-project.org/manuals.html}
\newline
\end{itemize}

Contributed documentation: \begin{itemize}
\item \url{http://cran.r-project.org/other-docs.html}
\newline
\end{itemize}
\end{block}

\end{frame}

\begin{frame}
\frametitle{R in the news}
R consortium announced:
\begin{itemize}
\item \tiny \url{https://www.r-consortium.org/}
\newline
\end{itemize}

Plans to implement R into SQL:
\begin{itemize}
\item \tiny \url{http://blog.revolutionanalytics.com/2015/05/r-in-sql-server.html}
\newline
\end{itemize}

Microsoft buys Revolution Analytics:
\begin{itemize}
\item \tiny \url{http://www.wired.com/2015/01/microsoft-acquires-open-source-data-science-company-revolution-analytics/}
\newline
\end{itemize}

\end{frame}

\begin{frame}
\frametitle{Integrating R into your Workflow}
You could start with using R for graphics...

\end{frame}



\begin{frame}
\frametitle{R Academia}
\begin{block}{Conferences about R:}
\textbf{useR! 2015} Aalborg, Denmark: \begin{itemize}
 \item http://user2015.math.aau.dk// 
 \newline
 \end{itemize}

\textbf{earl2015} London \& Boston: \begin{itemize}
\item \url{http://www.earl-conference.com/}
\newline
\end{itemize}
\end{block}

\begin{block}{A journal about R:}
\begin{itemize}
\item http://journal.r-project.org/
\end{itemize}
\end{block}
\end{frame}



\end{document}


